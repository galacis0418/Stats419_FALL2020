% Options for packages loaded elsewhere
\PassOptionsToPackage{unicode}{hyperref}
\PassOptionsToPackage{hyphens}{url}
%
\documentclass[
]{article}
\usepackage{lmodern}
\usepackage{amssymb,amsmath}
\usepackage{ifxetex,ifluatex}
\ifnum 0\ifxetex 1\fi\ifluatex 1\fi=0 % if pdftex
  \usepackage[T1]{fontenc}
  \usepackage[utf8]{inputenc}
  \usepackage{textcomp} % provide euro and other symbols
\else % if luatex or xetex
  \usepackage{unicode-math}
  \defaultfontfeatures{Scale=MatchLowercase}
  \defaultfontfeatures[\rmfamily]{Ligatures=TeX,Scale=1}
\fi
% Use upquote if available, for straight quotes in verbatim environments
\IfFileExists{upquote.sty}{\usepackage{upquote}}{}
\IfFileExists{microtype.sty}{% use microtype if available
  \usepackage[]{microtype}
  \UseMicrotypeSet[protrusion]{basicmath} % disable protrusion for tt fonts
}{}
\makeatletter
\@ifundefined{KOMAClassName}{% if non-KOMA class
  \IfFileExists{parskip.sty}{%
    \usepackage{parskip}
  }{% else
    \setlength{\parindent}{0pt}
    \setlength{\parskip}{6pt plus 2pt minus 1pt}}
}{% if KOMA class
  \KOMAoptions{parskip=half}}
\makeatother
\usepackage{xcolor}
\IfFileExists{xurl.sty}{\usepackage{xurl}}{} % add URL line breaks if available
\IfFileExists{bookmark.sty}{\usepackage{bookmark}}{\usepackage{hyperref}}
\hypersetup{
  pdftitle={R Notebook: learn chinese with NLP},
  hidelinks,
  pdfcreator={LaTeX via pandoc}}
\urlstyle{same} % disable monospaced font for URLs
\usepackage[margin=1in]{geometry}
\usepackage{color}
\usepackage{fancyvrb}
\newcommand{\VerbBar}{|}
\newcommand{\VERB}{\Verb[commandchars=\\\{\}]}
\DefineVerbatimEnvironment{Highlighting}{Verbatim}{commandchars=\\\{\}}
% Add ',fontsize=\small' for more characters per line
\usepackage{framed}
\definecolor{shadecolor}{RGB}{248,248,248}
\newenvironment{Shaded}{\begin{snugshade}}{\end{snugshade}}
\newcommand{\AlertTok}[1]{\textcolor[rgb]{0.94,0.16,0.16}{#1}}
\newcommand{\AnnotationTok}[1]{\textcolor[rgb]{0.56,0.35,0.01}{\textbf{\textit{#1}}}}
\newcommand{\AttributeTok}[1]{\textcolor[rgb]{0.77,0.63,0.00}{#1}}
\newcommand{\BaseNTok}[1]{\textcolor[rgb]{0.00,0.00,0.81}{#1}}
\newcommand{\BuiltInTok}[1]{#1}
\newcommand{\CharTok}[1]{\textcolor[rgb]{0.31,0.60,0.02}{#1}}
\newcommand{\CommentTok}[1]{\textcolor[rgb]{0.56,0.35,0.01}{\textit{#1}}}
\newcommand{\CommentVarTok}[1]{\textcolor[rgb]{0.56,0.35,0.01}{\textbf{\textit{#1}}}}
\newcommand{\ConstantTok}[1]{\textcolor[rgb]{0.00,0.00,0.00}{#1}}
\newcommand{\ControlFlowTok}[1]{\textcolor[rgb]{0.13,0.29,0.53}{\textbf{#1}}}
\newcommand{\DataTypeTok}[1]{\textcolor[rgb]{0.13,0.29,0.53}{#1}}
\newcommand{\DecValTok}[1]{\textcolor[rgb]{0.00,0.00,0.81}{#1}}
\newcommand{\DocumentationTok}[1]{\textcolor[rgb]{0.56,0.35,0.01}{\textbf{\textit{#1}}}}
\newcommand{\ErrorTok}[1]{\textcolor[rgb]{0.64,0.00,0.00}{\textbf{#1}}}
\newcommand{\ExtensionTok}[1]{#1}
\newcommand{\FloatTok}[1]{\textcolor[rgb]{0.00,0.00,0.81}{#1}}
\newcommand{\FunctionTok}[1]{\textcolor[rgb]{0.00,0.00,0.00}{#1}}
\newcommand{\ImportTok}[1]{#1}
\newcommand{\InformationTok}[1]{\textcolor[rgb]{0.56,0.35,0.01}{\textbf{\textit{#1}}}}
\newcommand{\KeywordTok}[1]{\textcolor[rgb]{0.13,0.29,0.53}{\textbf{#1}}}
\newcommand{\NormalTok}[1]{#1}
\newcommand{\OperatorTok}[1]{\textcolor[rgb]{0.81,0.36,0.00}{\textbf{#1}}}
\newcommand{\OtherTok}[1]{\textcolor[rgb]{0.56,0.35,0.01}{#1}}
\newcommand{\PreprocessorTok}[1]{\textcolor[rgb]{0.56,0.35,0.01}{\textit{#1}}}
\newcommand{\RegionMarkerTok}[1]{#1}
\newcommand{\SpecialCharTok}[1]{\textcolor[rgb]{0.00,0.00,0.00}{#1}}
\newcommand{\SpecialStringTok}[1]{\textcolor[rgb]{0.31,0.60,0.02}{#1}}
\newcommand{\StringTok}[1]{\textcolor[rgb]{0.31,0.60,0.02}{#1}}
\newcommand{\VariableTok}[1]{\textcolor[rgb]{0.00,0.00,0.00}{#1}}
\newcommand{\VerbatimStringTok}[1]{\textcolor[rgb]{0.31,0.60,0.02}{#1}}
\newcommand{\WarningTok}[1]{\textcolor[rgb]{0.56,0.35,0.01}{\textbf{\textit{#1}}}}
\usepackage{graphicx}
\makeatletter
\def\maxwidth{\ifdim\Gin@nat@width>\linewidth\linewidth\else\Gin@nat@width\fi}
\def\maxheight{\ifdim\Gin@nat@height>\textheight\textheight\else\Gin@nat@height\fi}
\makeatother
% Scale images if necessary, so that they will not overflow the page
% margins by default, and it is still possible to overwrite the defaults
% using explicit options in \includegraphics[width, height, ...]{}
\setkeys{Gin}{width=\maxwidth,height=\maxheight,keepaspectratio}
% Set default figure placement to htbp
\makeatletter
\def\fps@figure{htbp}
\makeatother
\setlength{\emergencystretch}{3em} % prevent overfull lines
\providecommand{\tightlist}{%
  \setlength{\itemsep}{0pt}\setlength{\parskip}{0pt}}
\setcounter{secnumdepth}{-\maxdimen} % remove section numbering

\title{R Notebook: learn chinese with NLP}
\author{}
\date{\vspace{-2.5em}}

\begin{document}
\maketitle

{
\setcounter{tocdepth}{4}
\tableofcontents
}
\begin{Shaded}
\begin{Highlighting}[]
\KeywordTok{library}\NormalTok{(devtools);}
\end{Highlighting}
\end{Shaded}

\begin{verbatim}
## Loading required package: usethis
\end{verbatim}

\begin{Shaded}
\begin{Highlighting}[]
\KeywordTok{library}\NormalTok{(humanVerseWSU);}

\NormalTok{path.github =}\StringTok{ "https://raw.githubusercontent.com/MonteShaffer/humanVerseWSU/master/"}\NormalTok{;}

\NormalTok{include.me =}\StringTok{ }\KeywordTok{paste0}\NormalTok{(path.github, }\StringTok{"misc/functions{-}nlp.R"}\NormalTok{);}
\CommentTok{\# source\_url( include.me );}
\NormalTok{include.me =}\StringTok{ }\KeywordTok{paste0}\NormalTok{(path.github, }\StringTok{"humanVerseWSU/R/functions{-}encryption.R"}\NormalTok{);}
\KeywordTok{source\_url}\NormalTok{( include.me );}
\end{Highlighting}
\end{Shaded}

\begin{verbatim}
## SHA-1 hash of file is da71dde620bed33db055778b752eefb476f7bf6b
\end{verbatim}

\begin{Shaded}
\begin{Highlighting}[]
\NormalTok{include.me =}\StringTok{ }\KeywordTok{paste0}\NormalTok{(path.github, }\StringTok{"misc/functions{-}chinese.R"}\NormalTok{);}
\CommentTok{\# source\_url( include.me );}


\NormalTok{path.to.nascent =}\StringTok{ "C:/Users/Alexander Nevsky/Dropbox/WSU{-}419/Fall 2020/\_\_student\_access\_\_/unit\_02\_confirmatory\_data\_analysis/nascent/"}\NormalTok{;}

\NormalTok{folder.nlp =}\StringTok{ "nlp/"}\NormalTok{;}
\NormalTok{path.to.nlp =}\StringTok{ }\KeywordTok{paste0}\NormalTok{(path.to.nascent, folder.nlp);}
\end{Highlighting}
\end{Shaded}

\hypertarget{chinese-word-translation-as-nlp-classifier-problem}{%
\section{Chinese word translation as NLP classifier
problem}\label{chinese-word-translation-as-nlp-classifier-problem}}

The Chinese language is ancient. It consists of glyphs that at times
have sub-glyphs or radicals.

Can we use a document training to translate a simple English word like
``water'' or ``ox''?

The mathematics we will use is the concept of set-theory. If ``water''
is in let's say ten records, and is unique to the other words in the
set, we can uniquely identify the matching Chinese glyph or glyphs.

\hypertarget{water-ox}{%
\subsection{1973 Water Ox}\label{water-ox}}

I was born in March 1973 \url{https://en.wikipedia.org/wiki/1973}

On the above page, it links to the Chinese Zodiac symbol ``Water Ox''
and provides the following symbols or glyphs \texttt{癸丑年}.

\hypertarget{decomposing-each-glyph}{%
\subsubsection{Decomposing each Glyph}\label{decomposing-each-glyph}}

There are some dictionaries that decompose each of these three glyphs,
for example
\url{https://hanzicraft.com/character/\%E7\%99\%B8\%E4\%B8\%91\%E5\%B9\%B4}

\begin{itemize}
\tightlist
\item
  Radical 1 (\texttt{癸}):
  \texttt{癸\ =\textgreater{}\ 癶\ (footsteps),\ 一\ (one),\ 大\ (big)}
\item
  Radical 2 (\texttt{丑}):
  \texttt{丑\ =\textgreater{}\ 刀\ (knife),\ 二\ (two)}
\item
  Radical 3 (\texttt{年}):
  \texttt{年\ =\textgreater{}\ 丿\ (bend),\ 一\ (one),\ 十\ (ten),\ ㇗\ (N/A),\ 丨\ (line)}
\end{itemize}

From the radical decomposition, I don't see ``Water'' or ``Ox''
anywhere. An important part of language decomposition is called
etymology: how were the words derived? This is a personal research
question I have.

\hypertarget{hanzicraft-resources}{%
\subsubsection{Hanzicraft Resources}\label{hanzicraft-resources}}

This website links to several ``internal'' dictionaries:

\begin{itemize}
\tightlist
\item
  \url{https://hanzicraft.com/lists/frequency}
\item
  \url{https://hanzicraft.com/lists/phonetic-sets}
\item
  \url{https://hanzicraft.com/lists/productive-component}
\end{itemize}

and to several ``external'' dictionaries
\url{https://hanzicraft.com/about}:

\begin{itemize}
\tightlist
\item
  \url{http://lwc.daanvanesch.nl/}
\item
  \url{https://cjkdecomp.codeplex.com/}
\item
  \url{http://www.mdbg.net/chindict/chindict.php?page=cc-cedict}
\item
  \url{http://lingua.mtsu.edu/chinese-computing/}
\end{itemize}

There are better ``linguistic data sets'', but for this notebook, we can
use this data as we see fit.

\hypertarget{translated-prose}{%
\subsection{Translated Prose}\label{translated-prose}}

Having single-word dictionaries has its use, but seeing the words in
context would provide other insights into definitions. So can I find a
dataset that has english and chinese from a known text?

\hypertarget{chinese-english-document}{%
\subsubsection{Chinese-English
Document}\label{chinese-english-document}}

Since the Bible is so pervasive, significant effort has been made in
translating it into many languages. In fact, the Gutenberg printing
press and much of Western Society is linked to the translation of the
Bible (King James created copyright law to prevent the dissemination of
un-approved bibles:
\url{https://en.wikipedia.org/wiki/Copyright\#Background} which led to
the modern U.S. law on copyright and patents).

\hypertarget{chinese-hgb-and-hb5}{%
\paragraph{Chinese HGB and HB5}\label{chinese-hgb-and-hb5}}

There is a website \url{http://www.o-bible.com/} that contains the Bible
in two variants of Chinese. One called ``GB'' and ``Big5''. For example,
Genesis 1:

\begin{itemize}
\tightlist
\item
  English KJV
  \url{http://www.o-bible.com/cgibin/ob.cgi?version=kjv\&book=gen\&chapter=1}
\item
  English BBE
  \url{http://www.o-bible.com/cgibin/ob.cgi?version=bbe\&book=ge\&chapter=1}
\item
  Chinese HGB
  \url{http://www.o-bible.com/cgibin/ob.cgi?version=hgb\&book=gen\&chapter=1}
\item
  Chinese HB5
  \url{http://www.o-bible.com/cgibin/ob.cgi?version=hb5\&book=gen\&chapter=1}
\end{itemize}

These versions seem to be very common (see
\url{http://web.mit.edu/jywang/www/cef/Bible/Chinese/hgb.htm}).

\begin{itemize}
\tightlist
\item
  GB refers to the ``national standard'' encoding and often uses the new
  simplified characters (tied to the current Chinese government:
  \url{https://en.wikipedia.org/wiki/Simplified_Chinese_characters}).
\item
  Big5 refers to the ``Taiwan/Hong Kong'' encoding and often uses
  traditional characters.
\end{itemize}

\hypertarget{chinese-punctuation}{%
\paragraph{Chinese Punctuation}\label{chinese-punctuation}}

This is a nice example because we can decompose text into books,
chapters, and verses
\url{https://www.gotquestions.org/divided-Bible-chapters-verses.html}
(paragraph-like). And then (within a verse) we can analyze each
sentence. The period ``.'' appears to be \texttt{。} and the comma ``,''
appears to be \texttt{、} or \texttt{,} and the colon ``:'' appears to
be \texttt{.} (Genesis 1:14 of HB5)

This will test the capabilities of R, as encoding other languages is a
bit of a challenge. We are also testing a large dataset than the
previous examples. More text to process.

\hypertarget{four-versions-2-in-english-1-in-each-form-of-chinese}{%
\paragraph{Four versions (2 in English, 1 in each form of
Chinese)}\label{four-versions-2-in-english-1-in-each-form-of-chinese}}

\hypertarget{read-in-the-data}{%
\subparagraph{Read in the data}\label{read-in-the-data}}

Each of the four versions have the same number of lines. Line 1 is a
header, and we can skip it. The keys in the KJV are slightly different,
so we will process that one last, and use the keys from BBE for all four
versions. I choose BBE because it is English and I can read it.

\begin{Shaded}
\begin{Highlighting}[]
\NormalTok{versions =}\StringTok{ }\KeywordTok{c}\NormalTok{(}\StringTok{"bbe"}\NormalTok{,}\StringTok{"kjv"}\NormalTok{,}\StringTok{"hb5"}\NormalTok{,}\StringTok{"hbg"}\NormalTok{);}
\NormalTok{langs    =}\StringTok{ }\KeywordTok{c}\NormalTok{(}\StringTok{"en{-}us"}\NormalTok{, }\StringTok{"en{-}gb"}\NormalTok{, }\StringTok{"zh{-}tw"}\NormalTok{, }\StringTok{"zh{-}cn"}\NormalTok{);}
\CommentTok{\# https://en.wikipedia.org/wiki/ISO\_639{-}1}
\CommentTok{\# https://en.wikipedia.org/wiki/ISO\_3166{-}1\_alpha{-}2}
\NormalTok{langK    =}\StringTok{ }\KeywordTok{c}\NormalTok{(}\StringTok{"American English"}\NormalTok{, }\StringTok{"English"}\NormalTok{, }\StringTok{"Traditional Chinese"}\NormalTok{, }\StringTok{"Simplified Chinese"}\NormalTok{);}
\CommentTok{\# we will use the standard codes ...}


\CommentTok{\# dataframe format}
\CommentTok{\# version .. lang .. book.n ... chap.n ... para.n (verse) ... para.text (verse)}

\NormalTok{path.to.obible =}\StringTok{ }\KeywordTok{paste0}\NormalTok{(path.to.nlp, }\StringTok{"\_data\_/chinese/o{-}bible/"}\NormalTok{);}
\end{Highlighting}
\end{Shaded}

\begin{Shaded}
\begin{Highlighting}[]
\CommentTok{\# https://cran.r{-}project.org/web/packages/corpus/vignettes/chinese.html}
\NormalTok{cstops \textless{}{-}}\StringTok{ "https://raw.githubusercontent.com/ropensci/textworkshop17/master/demos/chineseDemo/ChineseStopWords.txt"}
\NormalTok{csw \textless{}{-}}\StringTok{ }\KeywordTok{paste}\NormalTok{(}\KeywordTok{readLines}\NormalTok{(cstops, }\DataTypeTok{encoding =} \StringTok{"UTF{-}8"}\NormalTok{), }\DataTypeTok{collapse =} \StringTok{"}\CharTok{\textbackslash{}n}\StringTok{"}\NormalTok{) }\CommentTok{\# download}
\end{Highlighting}
\end{Shaded}


\end{document}
